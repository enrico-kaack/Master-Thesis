The rise of information technology with computers over the last decades allowed humanity to process data and solve problems in unprecedented performance. Computers require instructions in order to fulfil their purpose. The common language between humans and computers are programming languages. Software developers use programming languages to write source code that instructs computers to process data and solve problems. While a machine only follows the instructions described by the source code, the humans have to understand the purpose of the instructions. 

Nowadays, software becomes a vast market with a revenue of over 450 billion U.S. dollars in 2019~\cite{statista_software_nodate,costello_gartner_2019}. Creating new code often requires to read and understand preexisting code. As a consequence, it is essential to make code as readable and understandable as possible. 

The objectives of the clean code principles are a high level of readability and understandability of source code. Thus, the clean code principles comprise a set of rules and recommendations about writing code to be easy to read and understand. For computers, it is irrelevant whether or not a human can understand the code since they only execute the instructions. However, any developer who reads code that is written in accordance to the clean code principles profits from it. 

In \Cref{list:motivating_example}, the first listing shows an implementation of an algorithm without paying attention to clean code rules. In the second listing, the same algorithm is implemented following the clean code rules such as having one purpose per function or method calls in if-conditions. Executing the code will yield the same result. However, it is far simpler to understand the second than the first code. Reading the clean \texttt{get\_students\_eligible\_for\_exam} function let the developer immediately understand the two eligibility criteria for the exam. The comments like in the first example are not even necessary, since the code is self-describing.
Consequently, using or modifying the lower code is much easier and time-saving.

\begin{figure}[h]
\begin{tabular}{p{\textwidth}}
\begin{minipage}{1\textwidth}
\centering
\begin{lstlisting}[basicstyle=\tiny, language=Python, label=lst:tokens_1, caption={}]
class Student():
    def __init__(self, name):
        self.name = name
        self.exercises = []

def get_students_eligible_for_exam(students):
    #filter for students with enough points
    enough_points = []
    for student in students:
      points = 0
      for e in student.exercises:
        points += e
      if points >= 60:
        enough_points.append(student)
    
    #only students with at least 8 completed exercise
    eligible = []
    for student in enough_points:
      if len(student.exercises) >= 8:
        eligible.append(student)
    return eligible
\end{lstlisting}
\end{minipage}
\\
\begin{minipage}[c]{1\textwidth}
\centering
\begin{lstlisting}[basicstyle=\tiny, language=Python, label=lst:tokens_2, caption={}]
class Student():
    def __init__(self, name):
        self.name = name
        self.exercises = []
    
class ExamEligibilityChecker():
    MIN_NUMBER_OF_POINTS = 60
    NUMBER_OF_SOLVED_EXERCISES = 8
    
    def has_student_enough_points(student):
        total_points = 0
        for exercise in student.exercises:
            total_points += exercise
        return total_points >= ExamEligibilityChecker.MIN_NUMBER_OF_POINTS
    
    def has_student_enough_completed_exercises(student):
        return len(student.exercises) >= ExamEligibilityChecker.NUMBER_OF_SOLVED_EXERCISES

def get_students_eligible_for_exam(students):
    eligible_students = []
        for student in students:
        if ExamEligibilityChecker.has_student_enough_points(student) 
        and ExamEligibilityChecker.has_student_enough_completed_exercises(student):
            eligible_students.append(student)
    return eligible_students    
\end{lstlisting}
\end{minipage}
\end{tabular}
\caption{Motivating example code that checks, if a student is eligible for an exam. The first listing shows a smaller, more chaotic version of a possible implementation. The second example applies some clean code rules like a single purpose per function or method calls in if-conditions. Further clean code improvements are possible.}
\label{list:motivating_example}
\end{figure}


An automated tool for checking violations of the clean code rule can preserve the readability and understandability of the code for future modifications. Integrated into a build pipeline, it could reject code changes if necessary and act as a quality gate.
Furthermore, it can teach students about clean code rules. In practical works, an automated checker could assess the students' code and recommend improvements. The students can change the code and get immediate feedback if they bettered or worsen the code.


\section{Objectives and Contributions}
This thesis focus on the automated detection of clean code violations in Python source code. 
We have the following main objectives:
\begin{itemize}
    \item Design and implement a platform that can be extended with different automated checkers.
    \item Compare machine learning models on detecting code patterns that violate the clean code rules.
    \item Evaluate the generalisation capabilities to unseen pattern variations.  
\end{itemize}

To fulfil the objectives, we made the following contributions:
\begin{itemize}
    \item Overview over code quality, clean code rules, quantitative metrics and existing tools.
    \item Proposal for a taxonomy of the implementation complexity of automated checker for clean code violations.
    \item Design and implementation of the clean code analysis platform (CCAP) with focus on extensibility, useability and integration capabilities.
    \item Extension of CCAP with checkers for returning \texttt{None} and using direct comparisons in conditionals.
    \item Comparison of our CCAP with preexisting tools.
    \item Creation of a dataset and evaluation of a support-vector classifier, random forest classifier, gradient boosting classifier and an LSTM-based neural network on detecting problematic code patterns for that we may not find a deterministic checker.
    \item Introduction of an automated way to manipulate the code to contain unseen problem variations.
    \item Simulation of the impact of having a hand-collected dataset on the machine-learning models if the samples do not contain all possible variations of a problem pattern. We give ideas on how to handle the situation in a time-optimising way.
\end{itemize}

Based on our main objectives, we will evaluate the following research question:
\begin{description}
    \setlength{\itemsep}{1pt}
    \item[RQ1]What is the utility of the CCAP besides existing tools? 
    \item[RQ2]How do different machine learning models compare on the task of detecting non-clean code?
    \item[RQ3]Do machine-learning-based models cover a larger variety of cases than rule-based checker? 
\end{description}


\section{Structure}
This thesis is organised as follows: In \Cref{chap:background} we introduce code quality and clean code principles. We then classify the clean code principles depending on the implementation complexity of an automated checker. We introduce several quantitative metrics to measure code quality and provide an overview of existing tools that can determine the code quality and clean code violations to an extend. 
In \Cref{chap:approach}, we describe our approach for the CCAP and the machine-learning-based clean code classification. We evaluate the CCAP and machine-learning models and answer the research questions in \Cref{chap:evaluation}. Finally, we conclude our work with a summary and outlook in \Cref{chap:conclusion}.